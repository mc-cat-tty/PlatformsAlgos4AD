\chapter{Electronic Control Units}

In this chapter we will cover ECU types, architectural organization of ECUs, Software Defined Vehicles, software and hardware safety designs.

\section{History}
The history of vehicles, as far as control units are concerned, can be roughly divided into 6 phases:
\begin{enumerate}
  \item <1970 - Mechanical era - Purely mechanical systems, no electronics on board
  \item 1970-1980 - Electronic era - ECUs manage fuel injection and other subsystems of the powertrain/brakes/throttle. See Bosch ???
  \item 1980-2000 - Digital - 
  \item 2000-now - ADAS
  \item 2010-now - AD
\end{enumerate}

\section{State of the Art}
Mobileye EyeQ1

\section{ECU Types}
ECUs range from simple one-task microcontrollers to advanced AD compute units.

Inner computation compnents characterize the type of ECU.
CPU-based architectures are the simplest one. They differentiate between single- and multi-core architectures.

See: NVIDIA Parker architecture, NVIDIA Drive PX2

Why two separate processing islands? ???

FPGAs enable to provide hardware implementations of algorithms, taking advantage of the instrinsic parallelizable nature of the algorithm.

See: ACU - Apollo Computing Unit - used by Beidou for autonomous parking

ASICs are the most specific hardware units used for computation right now. They can be categorized into VPUs (Vison Processing Units), TPUs (Tensor Processing Units) and NPUs (Neural Processing Units).

Tesla HW3: TODO.
Tesla developed an architecture called FSD Computer 2 inside a SoC mounted on Tesla hardware design 4.

See: Hailo 8.

\section{Architectures}
Depending on the historical moment in which the architecture has been developed, different organizations of ECUs can be found.

In a distributed E/E architecture a network of dedicated ECUs is present in the vehicle.
This is the simplest one but also the most vulnerable to vendor's locking.

The different units are allowed to intercommunicate through a shared bus, such as CAN.
A domain-based architecture groups the ECUs in domains depending on their task. The ECUs belonging to the same domain can communicate between them.
Domains are interconnected through a fronteir ECU.

Zonal architecture is the evolution of domain-based architecture. This arch saves wiring and hareness.
It relies on the principle that a compute unit of each zone can process all the sensors and actuators of the zone.

The most recent architecture that is catching on is called Software Defined Vehicle enabled by a zonal E/E arch which allows to host heavy-duty algorithms, high-performance ECUs, virtualization, real-time connectivity.

View: rival zonal architecture.

\section{Safety}
Safety can be ensured in several ways, mainly through redundancy, implemented at either software or hardware level.
Software-level redundancy exploits algorithmic diversity to compare and check.
Hardware-level redundacy .

As you may notice diversity is a key concept in redundancy. Provides robustness against systematic algorithmic errors.
Replication itself mitigates random faults which can still affect diverse executions.

Comparing results of software and hardware replicas through a comparator or a voting circuit ensure different units arrive to the same result.

Vedi: Mariani NVDIA