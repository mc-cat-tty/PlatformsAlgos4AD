\chapter{Kalman Filter}

See: \url{https://www.kalmanfilter.net/default.aspx}

\section{Intuition}
An oversimplified example would be a dot/object moving in 1D world with a costant speed.

The basic idea under the Kalman filter is state estimation (see also dead reckoning).
That is to say that the state (position, velocity, etc.) is predicted by the filter based on a model (eg. kinematic model) of the system under analysis.

Pipeline: prediction (also referred to as estimation), correction update and 

Let's assume that we wear the wrong pair of glasses (translated would be comparable to noisy sensor) while walking.
What could happen? given that our vision of the world is distorted, we can wrongly estimate our position and compensate the uncorrect information with other proprioceptive measurements.
The filter has to keep this into consideration: both the model and the sensors can drift, diverge and accumulate error.

Lastly, problems can arise if the model is used in unexpected scenarios, such as a slippery road in case of a car.

TODO: diagram inti predict correct

From an historical point of view, USA and Russia were in the space race.

\section{Gaussians}
The uncertainty distribution of both state estimation and sensors measurement is modeled by a gaussian distribution.
In other words, each sensor has a gaussian distribution associted with the measurements it provides; the same holds for the predicted state, depending on how reliable the model is.

The correct-update cycle can be implemented in different several ways: continous prediction and once-every-period update, one prediction after each update update.

\subsection{Update and Prediction Fomulas}
TODO: see conditional probability associated with gaussian probability

\section{Multivariate KF}
Real-world problems are usually modeled through a multivariate Kalman Filter.
Multiple variables, aka multiple sensors, are fed into the state estimation block.

This time we not only want to estimate the system's state with an uncertain (to a certain amount) measurement, but also to extrapolate the covariance between sensors by observing how they dift over time with respect to each other.

TODO: KF formulas hidden variables

It looks like periodic updates are better than aperiodic ones.