\chapter{Clustering}
The objective of clustering is find and segment individual objects from a scene captured by a point cloud;
the objects usually lay on a plane that must be desiderably excluded from the cloud. 

The steps required to achieve clustering are:
\begin{itemize}
  \item down sample the PC to reduce points number for ease of processing
  \item segment and remove the ground plane
  \item create KD on top of the points
  \item compute euclidean distance between point pairs and build clusters
  \item the algorithm terminates when all points have been processed; aka when each points is part of a cluster
\end{itemize}

We are using the PCL library for the lab session.
PCD - Point Cloud Data - is the file format used to store point cloud data.
The header of this file lists some metadata related to the point-cloud.

\section{Filtering}
Voxel filtering is the technique that allows us to downsample 2D or 3D points.
It consists in computing the centroid of each voxel/2D square of the pointcloud.

RANSAC is a Random Consensus Algorithm for filtering 2D and 3D scenes by removing outliers.

\section{KD Tree}
TODO: Intuition, algorithms and visual examples

PCL Library offers different important methods and data structures:
- PointXYZ represents a point in a 3D space
- EuclideanClusterExtraction is the class that holds parameters and can be used to get the vector of clusters
- setClusterTolerance defines the size of spherical neuighbours used to cluster the cloud
- min cluster size is useful to filter out noise from the pointcloud and depends on the downsampling factor as well as the tolerance

